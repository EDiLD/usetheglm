\documentclass{scrartcl}\usepackage[]{graphicx}\usepackage[]{color}
%% maxwidth is the original width if it is less than linewidth
%% otherwise use linewidth (to make sure the graphics do not exceed the margin)
\makeatletter
\def\maxwidth{ %
  \ifdim\Gin@nat@width>\linewidth
    \linewidth
  \else
    \Gin@nat@width
  \fi
}
\makeatother

\definecolor{fgcolor}{rgb}{0.345, 0.345, 0.345}
\newcommand{\hlnum}[1]{\textcolor[rgb]{0.686,0.059,0.569}{#1}}%
\newcommand{\hlstr}[1]{\textcolor[rgb]{0.192,0.494,0.8}{#1}}%
\newcommand{\hlcom}[1]{\textcolor[rgb]{0.678,0.584,0.686}{\textit{#1}}}%
\newcommand{\hlopt}[1]{\textcolor[rgb]{0,0,0}{#1}}%
\newcommand{\hlstd}[1]{\textcolor[rgb]{0.345,0.345,0.345}{#1}}%
\newcommand{\hlkwa}[1]{\textcolor[rgb]{0.161,0.373,0.58}{\textbf{#1}}}%
\newcommand{\hlkwb}[1]{\textcolor[rgb]{0.69,0.353,0.396}{#1}}%
\newcommand{\hlkwc}[1]{\textcolor[rgb]{0.333,0.667,0.333}{#1}}%
\newcommand{\hlkwd}[1]{\textcolor[rgb]{0.737,0.353,0.396}{\textbf{#1}}}%

\usepackage{framed}
\makeatletter
\newenvironment{kframe}{%
 \def\at@end@of@kframe{}%
 \ifinner\ifhmode%
  \def\at@end@of@kframe{\end{minipage}}%
  \begin{minipage}{\columnwidth}%
 \fi\fi%
 \def\FrameCommand##1{\hskip\@totalleftmargin \hskip-\fboxsep
 \colorbox{shadecolor}{##1}\hskip-\fboxsep
     % There is no \\@totalrightmargin, so:
     \hskip-\linewidth \hskip-\@totalleftmargin \hskip\columnwidth}%
 \MakeFramed {\advance\hsize-\width
   \@totalleftmargin\z@ \linewidth\hsize
   \@setminipage}}%
 {\par\unskip\endMakeFramed%
 \at@end@of@kframe}
\makeatother

\definecolor{shadecolor}{rgb}{.97, .97, .97}
\definecolor{messagecolor}{rgb}{0, 0, 0}
\definecolor{warningcolor}{rgb}{1, 0, 1}
\definecolor{errorcolor}{rgb}{1, 0, 0}
\newenvironment{knitrout}{}{} % an empty environment to be redefined in TeX

\usepackage{alltt}

\usepackage[utf8]{inputenc}
\usepackage{graphicx}%GRaphiken
\usepackage{tabularx}%Tabellen!
\usepackage[english]{babel}% Zeilentrennung besser
\usepackage{url}% Urls besser
\usepackage{textcomp}% Sonderzeichen
\usepackage{amsmath}%maths / equations
\usepackage{helvet}% Schrift Helvetica
% \usepackage[helvet]{sfmath}% Helvet also in Math modes
% \renewcommand\familydefault{\sfdefault}
\usepackage{sansmath} % sans in math
\usepackage{float}
\usepackage{todonotes}
\usepackage[
	left=3cm,
	right=2cm,
	top=1.5cm,
	bottom=1cm
	,
	includeheadfoot
	]{geometry}														% Satzspiegel
\usepackage[
	round,	%(defaultage in the main file and \input ) for round parentheses;
	%square,	% for square brackets;
	%curly,	% for curly braces;
	%angle,	% for angle brackets;
	colon,	% (default) to separate multiple citations with colons;
	%comma,	% to use commas as separaters;
	authoryear,% (default) for author-year citations;
	%numbers,	% for numerical citations;
	%super,	% for superscripted numerical citations, as in Nature;
	sort,		% orders multiple citations into the sequence in which they appear in the list of 				references;
	%sort&compress,    % as sort but in addition multiple numerical citations
                   % are compressed if possible (as 3-6, 15);
	%longnamesfirst,  % makes the first citation of any reference the equivalent of
                   % the starred variant (full author list) and subsequent citations
                   %normal (abbreviated list);
	%sectionbib,      % redefines \thebibliography to issue \section* instead of \chapter*;
                   % valid only for classes with a \chapter command;
                   % to be used with the chapterbib package;
	%nonamebreak,     % keeps all the authors names in a citation on one line;
                   %causes overfull hboxes but helps with some hyperref problems.
]{natbib}											    			% Literaturverzeichnis
\usepackage{scrhack}   % kills \float@addtolists!  warning
\usepackage[pdfpagelabels,plainpages=false, pageanchor=false]{hyperref}	


%% andere Einstellungen
\linespread{1.5}% 1.5 Zeilenabstand			
\graphicspath{{fig/}}                     % path to graphics

%% ----------------------------------------------------------------------------
\title{Ecotoxicology is not normal.}
\subtitle{How the use of proper statistical models can increase statistical power in ecotoxicological experiments.}
\author{Eduard Szöcs, Ralf B. Schäfer}
\date{\today}


% ----------------------------------------------------------------------------
\IfFileExists{upquote.sty}{\usepackage{upquote}}{}
\begin{document}
\maketitle
\section{Supplement 1 - Additional Figures / Tables}

\subsection{Count data simulations}
% latex table generated in R 3.1.2 by xtable 1.7-4 package
% Mon Feb  9 15:33:12 2015
\begin{table}[H]
\centering
\caption{Count data simulations - Proportion of models converged. N = sample sizes, 
             $\mu_C$ = mean abundance in control, LM = Linear model after transformation, 
             $GLM_{nb}$ = negative binomial model, $GLM_{qp}$ = quasi-Poisson model.} 
\label{tab:conv}
{\footnotesize
\begin{tabular}{rrrrr}
  \hline
N & $\mu_C$ & LM & $GLM_{nb}$ & $GLM_{qp}$ \\ 
  \hline
3.00 & 2.00 & 1.00 & 0.30 & 1.00 \\ 
  3.00 & 4.00 & 1.00 & 0.51 & 1.00 \\ 
  3.00 & 8.00 & 1.00 & 0.72 & 1.00 \\ 
  3.00 & 16.00 & 1.00 & 0.93 & 1.00 \\ 
  3.00 & 32.00 & 1.00 & 0.98 & 1.00 \\ 
  3.00 & 64.00 & 1.00 & 1.00 & 1.00 \\ 
  3.00 & 128.00 & 1.00 & 1.00 & 1.00 \\ 
  6.00 & 2.00 & 1.00 & 0.57 & 1.00 \\ 
  6.00 & 4.00 & 1.00 & 0.87 & 1.00 \\ 
  6.00 & 8.00 & 1.00 & 0.97 & 1.00 \\ 
  6.00 & 16.00 & 1.00 & 1.00 & 1.00 \\ 
  6.00 & 32.00 & 1.00 & 1.00 & 1.00 \\ 
  6.00 & 64.00 & 1.00 & 1.00 & 1.00 \\ 
  6.00 & 128.00 & 1.00 & 1.00 & 1.00 \\ 
  9.00 & 2.00 & 1.00 & 0.82 & 1.00 \\ 
  9.00 & 4.00 & 1.00 & 0.98 & 1.00 \\ 
  9.00 & 8.00 & 1.00 & 0.99 & 1.00 \\ 
  9.00 & 16.00 & 1.00 & 1.00 & 1.00 \\ 
  9.00 & 32.00 & 1.00 & 1.00 & 1.00 \\ 
  9.00 & 64.00 & 1.00 & 1.00 & 1.00 \\ 
  9.00 & 128.00 & 1.00 & 1.00 & 1.00 \\ 
   \hline
\end{tabular}
}
\end{table}

% latex table generated in R 3.1.2 by xtable 1.7-4 package
% Wed Jan 21 15:32:38 2015
\begin{table}[H]
\centering
\caption{Count data simulations - Power to detect a global treatment effect. N = sample sizes, 
             $\mu_C$ = mean abundance in control, LM = Linear model after transformation, 
             $GLM_{nb}$ = negative binomial model, $GLM_{qp}$ = quasi-Poisson model, 
             $GLM_{pb}$ = negative binomial model with parametric boostrap, np = Kruskal-Wallis test.} 
\label{tab:pow_glob_c}
{\footnotesize
\begin{tabular}{rrrrrrr}
  \hline
N & $\mu_C$ & LM & $GLM_{nb}$ & $GLM_{qp}$ & $GLM_{pb}$ & np \\ 
  \hline
3.00 & 2.00 & 0.14 & 0.17 & 0.19 & 0.07 & 0.09 \\ 
  3.00 & 4.00 & 0.13 & 0.18 & 0.20 & 0.08 & 0.05 \\ 
  3.00 & 8.00 & 0.21 & 0.38 & 0.24 & 0.19 & 0.12 \\ 
  3.00 & 16.00 & 0.28 & 0.45 & 0.32 & 0.26 & 0.18 \\ 
  3.00 & 32.00 & 0.33 & 0.54 & 0.45 & 0.36 & 0.18 \\ 
  3.00 & 64.00 & 0.30 & 0.57 & 0.35 & 0.37 & 0.14 \\ 
  3.00 & 128.00 & 0.25 & 0.57 & 0.35 & 0.32 & 0.13 \\ 
  6.00 & 2.00 & 0.30 & 0.33 & 0.33 & 0.21 & 0.27 \\ 
  6.00 & 4.00 & 0.36 & 0.45 & 0.43 & 0.33 & 0.26 \\ 
  6.00 & 8.00 & 0.44 & 0.65 & 0.59 & 0.53 & 0.44 \\ 
  6.00 & 16.00 & 0.58 & 0.78 & 0.72 & 0.65 & 0.49 \\ 
  6.00 & 32.00 & 0.59 & 0.82 & 0.71 & 0.67 & 0.51 \\ 
  6.00 & 64.00 & 0.65 & 0.74 & 0.73 & 0.68 & 0.63 \\ 
  6.00 & 128.00 & 0.80 & 0.91 & 0.85 & 0.84 & 0.70 \\ 
  9.00 & 2.00 & 0.34 & 0.30 & 0.35 & 0.27 & 0.30 \\ 
  9.00 & 4.00 & 0.54 & 0.65 & 0.65 & 0.61 & 0.47 \\ 
  9.00 & 8.00 & 0.56 & 0.74 & 0.73 & 0.67 & 0.58 \\ 
  9.00 & 16.00 & 0.80 & 0.89 & 0.90 & 0.88 & 0.79 \\ 
  9.00 & 32.00 & 0.88 & 0.93 & 0.92 & 0.91 & 0.89 \\ 
  9.00 & 64.00 & 0.90 & 0.94 & 0.95 & 0.93 & 0.91 \\ 
  9.00 & 128.00 & 0.91 & 0.95 & 0.93 & 0.94 & 0.91 \\ 
   \hline
\end{tabular}
}
\end{table}

% latex table generated in R 3.1.3 by xtable 1.7-4 package
% Mon Mar 23 13:00:23 2015
\begin{table}[H]
\centering
\caption{Count data simulations - Power to detect LOEC. N = sample sizes, 
             $\mu_C$ = mean abundance in control, LM = Linear model after transformation, 
             $GLM_{nb}$ = negative binomial model, $GLM_{qp}$ = quasi-Poisson model, 
            $GLM_{p}$ = Poisson model, np = pairwise Wilcoxon test.} 
\label{tab:pow_loec_c}
{\footnotesize
\begin{tabular}{rrrrrrr}
  \hline
N & $\mu_C$ & LM & $GLM_{nb}$ & $GLM_{qp}$ & $GLM_{p}$ & np \\ 
  \hline
3.00 & 2.00 & 0.05 & 0.01 & 0.02 & 0.02 & 0.00 \\ 
  3.00 & 4.00 & 0.08 & 0.09 & 0.08 & 0.15 & 0.00 \\ 
  3.00 & 8.00 & 0.11 & 0.22 & 0.12 & 0.30 & 0.00 \\ 
  3.00 & 16.00 & 0.13 & 0.30 & 0.18 & 0.42 & 0.00 \\ 
  3.00 & 32.00 & 0.17 & 0.35 & 0.22 & 0.50 & 0.00 \\ 
  3.00 & 64.00 & 0.19 & 0.37 & 0.23 & 0.51 & 0.00 \\ 
  3.00 & 128.00 & 0.18 & 0.37 & 0.23 & 0.53 & 0.00 \\ 
  6.00 & 2.00 & 0.14 & 0.11 & 0.09 & 0.15 & 0.06 \\ 
  6.00 & 4.00 & 0.17 & 0.23 & 0.19 & 0.30 & 0.12 \\ 
  6.00 & 8.00 & 0.28 & 0.39 & 0.32 & 0.52 & 0.20 \\ 
  6.00 & 16.00 & 0.33 & 0.48 & 0.39 & 0.59 & 0.23 \\ 
  6.00 & 32.00 & 0.40 & 0.54 & 0.47 & 0.64 & 0.28 \\ 
  6.00 & 64.00 & 0.44 & 0.56 & 0.48 & 0.61 & 0.29 \\ 
  6.00 & 128.00 & 0.44 & 0.57 & 0.49 & 0.56 & 0.29 \\ 
  9.00 & 2.00 & 0.19 & 0.20 & 0.18 & 0.26 & 0.13 \\ 
  9.00 & 4.00 & 0.29 & 0.37 & 0.31 & 0.48 & 0.27 \\ 
  9.00 & 8.00 & 0.40 & 0.52 & 0.46 & 0.62 & 0.35 \\ 
  9.00 & 16.00 & 0.51 & 0.63 & 0.57 & 0.70 & 0.45 \\ 
  9.00 & 32.00 & 0.57 & 0.69 & 0.63 & 0.68 & 0.52 \\ 
  9.00 & 64.00 & 0.61 & 0.72 & 0.66 & 0.65 & 0.53 \\ 
  9.00 & 128.00 & 0.65 & 0.73 & 0.68 & 0.61 & 0.58 \\ 
   \hline
\end{tabular}
}
\end{table}

% latex table generated in R 3.1.2 by xtable 1.7-4 package
% Wed Jan 21 15:32:38 2015
\begin{table}[H]
\centering
\caption{Count data simulations - Type 1 error to detect a global treatment effect. N = sample sizes, 
             $\mu_C$ = mean abundance in control, LM = Linear model after transformation, 
             $GLM_{nb}$ = negative binomial model, $GLM_{qp}$ = quasi-Poisson model, 
             $GLM_{pb}$ = negative binomial model with parametric boostrap, np = Kruskal-Wallis test.} 
\label{tab:t1_glob_c}
{\footnotesize
\begin{tabular}{rrrrrrr}
  \hline
N & $\mu_C$ & LM & $GLM_{nb}$ & $GLM_{qp}$ & $GLM_{pb}$ & np \\ 
  \hline
3.00 & 2.00 & 0.09 & 0.03 & 0.00 & 0.10 & 0.04 \\ 
  3.00 & 4.00 & 0.07 & 0.11 & 0.04 & 0.06 & 0.03 \\ 
  3.00 & 8.00 & 0.05 & 0.11 & 0.09 & 0.07 & 0.01 \\ 
  3.00 & 16.00 & 0.03 & 0.12 & 0.05 & 0.03 & 0.01 \\ 
  3.00 & 32.00 & 0.05 & 0.14 & 0.05 & 0.04 & 0.00 \\ 
  3.00 & 64.00 & 0.02 & 0.11 & 0.04 & 0.04 & 0.00 \\ 
  3.00 & 128.00 & 0.07 & 0.19 & 0.05 & 0.09 & 0.02 \\ 
  6.00 & 2.00 & 0.04 & 0.03 & 0.03 & 0.05 & 0.02 \\ 
  6.00 & 4.00 & 0.04 & 0.12 & 0.05 & 0.09 & 0.04 \\ 
  6.00 & 8.00 & 0.05 & 0.04 & 0.04 & 0.04 & 0.03 \\ 
  6.00 & 16.00 & 0.04 & 0.09 & 0.04 & 0.06 & 0.04 \\ 
  6.00 & 32.00 & 0.06 & 0.08 & 0.06 & 0.07 & 0.05 \\ 
  6.00 & 64.00 & 0.05 & 0.06 & 0.05 & 0.05 & 0.03 \\ 
  6.00 & 128.00 & 0.04 & 0.09 & 0.02 & 0.04 & 0.01 \\ 
  9.00 & 2.00 & 0.04 & 0.03 & 0.03 & 0.05 & 0.05 \\ 
  9.00 & 4.00 & 0.05 & 0.07 & 0.03 & 0.07 & 0.04 \\ 
  9.00 & 8.00 & 0.08 & 0.12 & 0.07 & 0.09 & 0.08 \\ 
  9.00 & 16.00 & 0.07 & 0.09 & 0.06 & 0.08 & 0.06 \\ 
  9.00 & 32.00 & 0.06 & 0.07 & 0.05 & 0.06 & 0.05 \\ 
  9.00 & 64.00 & 0.03 & 0.06 & 0.04 & 0.04 & 0.03 \\ 
  9.00 & 128.00 & 0.05 & 0.07 & 0.04 & 0.07 & 0.02 \\ 
   \hline
\end{tabular}
}
\end{table}

% latex table generated in R 3.1.3 by xtable 1.7-4 package
% Mon Mar 23 13:00:27 2015
\begin{table}[H]
\centering
\caption{Count data simulations - Type 1 error to detect LOEC. N = sample sizes, 
             $\mu_C$ = mean abundance in control, LM = Linear model after transformation, 
             $GLM_{nb}$ = negative binomial model, $GLM_{qp}$ = quasi-Poisson model, 
            $GLM_{p}$ = Poisson model, np = pairwise Wilcoxon.} 
\label{tab:t1_loec_c}
{\footnotesize
\begin{tabular}{rrrrrrr}
  \hline
N & $\mu_C$ & LM & $GLM_{nb}$ & $GLM_{qp}$ & $GLM_{p}$ & np \\ 
  \hline
3.00 & 2.00 & 0.05 & 0.02 & 0.02 & 0.02 & 0.00 \\ 
  3.00 & 4.00 & 0.04 & 0.08 & 0.04 & 0.14 & 0.00 \\ 
  3.00 & 8.00 & 0.05 & 0.11 & 0.06 & 0.24 & 0.00 \\ 
  3.00 & 16.00 & 0.03 & 0.11 & 0.04 & 0.36 & 0.00 \\ 
  3.00 & 32.00 & 0.04 & 0.15 & 0.05 & 0.55 & 0.00 \\ 
  3.00 & 64.00 & 0.05 & 0.16 & 0.06 & 0.61 & 0.00 \\ 
  3.00 & 128.00 & 0.04 & 0.13 & 0.05 & 0.68 & 0.00 \\ 
  6.00 & 2.00 & 0.04 & 0.04 & 0.02 & 0.07 & 0.02 \\ 
  6.00 & 4.00 & 0.03 & 0.06 & 0.03 & 0.15 & 0.02 \\ 
  6.00 & 8.00 & 0.04 & 0.08 & 0.05 & 0.26 & 0.03 \\ 
  6.00 & 16.00 & 0.04 & 0.08 & 0.05 & 0.37 & 0.03 \\ 
  6.00 & 32.00 & 0.04 & 0.08 & 0.04 & 0.52 & 0.03 \\ 
  6.00 & 64.00 & 0.05 & 0.10 & 0.05 & 0.61 & 0.04 \\ 
  6.00 & 128.00 & 0.04 & 0.08 & 0.04 & 0.66 & 0.05 \\ 
  9.00 & 2.00 & 0.03 & 0.05 & 0.04 & 0.08 & 0.03 \\ 
  9.00 & 4.00 & 0.04 & 0.06 & 0.05 & 0.15 & 0.04 \\ 
  9.00 & 8.00 & 0.04 & 0.05 & 0.04 & 0.27 & 0.04 \\ 
  9.00 & 16.00 & 0.04 & 0.07 & 0.04 & 0.38 & 0.03 \\ 
  9.00 & 32.00 & 0.03 & 0.05 & 0.04 & 0.49 & 0.03 \\ 
  9.00 & 64.00 & 0.04 & 0.06 & 0.04 & 0.61 & 0.04 \\ 
  9.00 & 128.00 & 0.04 & 0.06 & 0.04 & 0.67 & 0.04 \\ 
   \hline
\end{tabular}
}
\end{table}


\subsection{Binomial data simulations}
% latex table generated in R 3.1.3 by xtable 1.7-4 package
% Mon Mar 16 16:45:12 2015
\begin{table}[H]
\centering
\caption{Binomial data simulations - Power to detect a global treatment effect. N = sample sizes, 
             $p_E$ = probability in effect treatments, LM = Linear model after transformation, 
             $GLM$ = binomial model, np = Kruskal-Wallis test.} 
\label{tab:pow_glob_p}
{\footnotesize
\begin{tabular}{rrrrr}
  \hline
N & $p_E$ & LM & $GLM$ & np \\ 
  \hline
3.00 & 0.60 & 0.95 & 1.00 & 0.86 \\ 
  3.00 & 0.65 & 0.87 & 0.99 & 0.73 \\ 
  3.00 & 0.70 & 0.78 & 0.97 & 0.64 \\ 
  3.00 & 0.75 & 0.61 & 0.85 & 0.44 \\ 
  3.00 & 0.80 & 0.42 & 0.63 & 0.28 \\ 
  3.00 & 0.85 & 0.21 & 0.42 & 0.10 \\ 
  3.00 & 0.90 & 0.09 & 0.13 & 0.04 \\ 
  3.00 & 0.95 & 0.06 & 0.07 & 0.03 \\ 
  6.00 & 0.60 & 1.00 & 1.00 & 1.00 \\ 
  6.00 & 0.65 & 1.00 & 1.00 & 1.00 \\ 
  6.00 & 0.70 & 1.00 & 1.00 & 1.00 \\ 
  6.00 & 0.75 & 0.98 & 1.00 & 0.96 \\ 
  6.00 & 0.80 & 0.83 & 0.88 & 0.80 \\ 
  6.00 & 0.85 & 0.55 & 0.64 & 0.50 \\ 
  6.00 & 0.90 & 0.18 & 0.24 & 0.14 \\ 
  6.00 & 0.95 & 0.04 & 0.06 & 0.02 \\ 
  9.00 & 0.60 & 1.00 & 1.00 & 1.00 \\ 
  9.00 & 0.65 & 1.00 & 1.00 & 1.00 \\ 
  9.00 & 0.70 & 1.00 & 1.00 & 1.00 \\ 
  9.00 & 0.75 & 1.00 & 1.00 & 1.00 \\ 
  9.00 & 0.80 & 0.98 & 0.99 & 0.97 \\ 
  9.00 & 0.85 & 0.76 & 0.83 & 0.73 \\ 
  9.00 & 0.90 & 0.28 & 0.32 & 0.25 \\ 
  9.00 & 0.95 & 0.06 & 0.06 & 0.05 \\ 
   \hline
\end{tabular}
}
\end{table}

% latex table generated in R 3.1.2 by xtable 1.7-4 package
% Mon Feb  9 15:33:20 2015
\begin{table}[H]
\centering
\caption{Count data simulations - Power to detect LOEC. N = sample sizes, 
             $p_E$ = probability in effect treatments, LM = Linear model after transformation, 
             $GLM$ = binomial model, np = pairwise Wilcoxon.} 
\label{tab:pow_loec_p}
{\footnotesize
\begin{tabular}{rrrrr}
  \hline
N & $p_E$ & LM & $GLM$ & np \\ 
  \hline
3.00 & 0.60 & 0.80 & 0.71 & 0.00 \\ 
  3.00 & 0.65 & 0.72 & 0.58 & 0.00 \\ 
  3.00 & 0.70 & 0.59 & 0.43 & 0.00 \\ 
  3.00 & 0.75 & 0.42 & 0.23 & 0.00 \\ 
  3.00 & 0.80 & 0.28 & 0.11 & 0.00 \\ 
  3.00 & 0.85 & 0.12 & 0.03 & 0.00 \\ 
  3.00 & 0.90 & 0.03 & 0.01 & 0.00 \\ 
  3.00 & 0.95 & 0.01 & 0.00 & 0.00 \\ 
  6.00 & 0.60 & 1.00 & 0.95 & 0.98 \\ 
  6.00 & 0.65 & 0.99 & 0.95 & 0.95 \\ 
  6.00 & 0.70 & 0.94 & 0.92 & 0.83 \\ 
  6.00 & 0.75 & 0.79 & 0.76 & 0.59 \\ 
  6.00 & 0.80 & 0.53 & 0.51 & 0.32 \\ 
  6.00 & 0.85 & 0.32 & 0.22 & 0.13 \\ 
  6.00 & 0.90 & 0.09 & 0.04 & 0.03 \\ 
  6.00 & 0.95 & 0.01 & 0.00 & 0.00 \\ 
  9.00 & 0.60 & 0.99 & 0.98 & 0.98 \\ 
  9.00 & 0.65 & 1.00 & 0.99 & 0.97 \\ 
  9.00 & 0.70 & 0.99 & 0.96 & 0.95 \\ 
  9.00 & 0.75 & 0.96 & 0.97 & 0.92 \\ 
  9.00 & 0.80 & 0.82 & 0.81 & 0.72 \\ 
  9.00 & 0.85 & 0.44 & 0.50 & 0.34 \\ 
  9.00 & 0.90 & 0.16 & 0.14 & 0.08 \\ 
  9.00 & 0.95 & 0.03 & 0.01 & 0.00 \\ 
   \hline
\end{tabular}
}
\end{table}

% latex table generated in R 3.1.2 by xtable 1.7-4 package
% Mon Feb  9 15:33:20 2015
\begin{table}[H]
\centering
\caption{Binomial data simulations - Type 1 error to detect a global treatment effect. N = sample sizes, 
             $p$ = probability, LM = Linear model after transformation, 
             $GLM$ = binomial model, np = Kruskal-Wallis test.} 
\label{tab:t1_glob_p}
{\footnotesize
\begin{tabular}{rrrrr}
  \hline
N & $p$ & LM & $GLM$ & np \\ 
  \hline
3.00 & 0.60 & 0.03 & 0.03 & 0.02 \\ 
  3.00 & 0.65 & 0.04 & 0.06 & 0.01 \\ 
  3.00 & 0.70 & 0.05 & 0.04 & 0.00 \\ 
  3.00 & 0.75 & 0.05 & 0.06 & 0.01 \\ 
  3.00 & 0.80 & 0.06 & 0.06 & 0.01 \\ 
  3.00 & 0.85 & 0.06 & 0.06 & 0.02 \\ 
  3.00 & 0.90 & 0.06 & 0.09 & 0.02 \\ 
  3.00 & 0.95 & 0.04 & 0.04 & 0.02 \\ 
  6.00 & 0.60 & 0.07 & 0.08 & 0.06 \\ 
  6.00 & 0.65 & 0.04 & 0.05 & 0.04 \\ 
  6.00 & 0.70 & 0.04 & 0.06 & 0.04 \\ 
  6.00 & 0.75 & 0.04 & 0.06 & 0.04 \\ 
  6.00 & 0.80 & 0.05 & 0.06 & 0.03 \\ 
  6.00 & 0.85 & 0.04 & 0.05 & 0.03 \\ 
  6.00 & 0.90 & 0.06 & 0.08 & 0.04 \\ 
  6.00 & 0.95 & 0.04 & 0.08 & 0.03 \\ 
  9.00 & 0.60 & 0.05 & 0.04 & 0.04 \\ 
  9.00 & 0.65 & 0.06 & 0.04 & 0.04 \\ 
  9.00 & 0.70 & 0.08 & 0.06 & 0.07 \\ 
  9.00 & 0.75 & 0.04 & 0.04 & 0.03 \\ 
  9.00 & 0.80 & 0.03 & 0.02 & 0.02 \\ 
  9.00 & 0.85 & 0.03 & 0.02 & 0.02 \\ 
  9.00 & 0.90 & 0.03 & 0.05 & 0.02 \\ 
  9.00 & 0.95 & 0.05 & 0.07 & 0.04 \\ 
   \hline
\end{tabular}
}
\end{table}

% latex table generated in R 3.1.3 by xtable 1.7-4 package
% Mon Mar 16 16:45:18 2015
\begin{table}[H]
\centering
\caption{Binomial data simulations - Type 1 error to detect LOEC. N = sample sizes, 
             $p$ = probability, LM = Linear model after transformation, 
             $GLM$ = binomial model, np = pairwise Wilcoxon.} 
\label{tab:t1_loec_p}
{\footnotesize
\begin{tabular}{rrrrr}
  \hline
N & $p_E$ & LM & $GLM$ & np \\ 
  \hline
3.00 & 0.60 & 0.03 & 0.04 & 0.00 \\ 
  3.00 & 0.65 & 0.04 & 0.04 & 0.00 \\ 
  3.00 & 0.70 & 0.05 & 0.04 & 0.00 \\ 
  3.00 & 0.75 & 0.05 & 0.02 & 0.00 \\ 
  3.00 & 0.80 & 0.08 & 0.06 & 0.00 \\ 
  3.00 & 0.85 & 0.05 & 0.04 & 0.00 \\ 
  3.00 & 0.90 & 0.06 & 0.00 & 0.00 \\ 
  3.00 & 0.95 & 0.06 & 0.00 & 0.00 \\ 
  6.00 & 0.60 & 0.06 & 0.04 & 0.01 \\ 
  6.00 & 0.65 & 0.02 & 0.04 & 0.02 \\ 
  6.00 & 0.70 & 0.05 & 0.05 & 0.02 \\ 
  6.00 & 0.75 & 0.04 & 0.06 & 0.04 \\ 
  6.00 & 0.80 & 0.04 & 0.04 & 0.02 \\ 
  6.00 & 0.85 & 0.06 & 0.04 & 0.02 \\ 
  6.00 & 0.90 & 0.06 & 0.03 & 0.02 \\ 
  6.00 & 0.95 & 0.05 & 0.00 & 0.01 \\ 
  9.00 & 0.60 & 0.06 & 0.06 & 0.04 \\ 
  9.00 & 0.65 & 0.04 & 0.03 & 0.02 \\ 
  9.00 & 0.70 & 0.06 & 0.06 & 0.05 \\ 
  9.00 & 0.75 & 0.05 & 0.07 & 0.04 \\ 
  9.00 & 0.80 & 0.03 & 0.03 & 0.01 \\ 
  9.00 & 0.85 & 0.05 & 0.04 & 0.02 \\ 
  9.00 & 0.90 & 0.03 & 0.03 & 0.02 \\ 
  9.00 & 0.95 & 0.04 & 0.01 & 0.01 \\ 
   \hline
\end{tabular}
}
\end{table}


\end{document} 
